\documentclass[main.tex]{subfiles}
\begin{document}
\chapter{Execution model for the \mppalong}
\thispagestyle{chapstyle}
\label{chap_execModel}
\minitoc

In this chapter, based on the detailed description of the \mppalong of
Chapter~\ref{chap_systemModel}, we provide mathematical expressions enabling
the computation of interference penalties when accessing shared resources such
as the local SRAM banks, the Network on Chip or the external DDR-SDRAM. We
assess the cost of an approach considering worst-case pathological competitors
to safely bound the WCET of an application independently from its co-runners.
Finally, we provide an execution model which mitigates this cost by restricting
the behaviour of applications when sharing resources.


\section{Bounds on shared resources access time}
\label{sec_execModel_boundsSharedRes}

In this section, we provide models of the shared resources such as the SRAM,
the NoC the DDR-SDRAM.  The idea is firstly to identify the possibilities of
bounding access time to those resources without making assumptions on the
applications behaviour; and secondly, to estimate the amount of time that could
be saved by reducing the access time margins by making some assumptions. The
models presented here are derived from the composition rules that we originally
presented in~\cite{Perret16}.


\subsection{SRAM bank access time} \label{ssec_execModel_SRAMaccessModel}
Accesses to SRAM banks in compute clusters are ruled by a complex arbitration
policy.  Fortunately, the access protocol for SRAM is simple and no refresh is
required. Assuming no DMA reception and a service time of $T_{access}^{SRAM} =
10 \times t_{ck}$ (with $t_{ck}$ the duration of a clock cycle) for a single
memory access of 8 bytes (9 cycles of latency and 1 cycle per double-word
consecutively fetched), the maximum duration of a memory transaction of
$S_{trans}$ bytes initiated by a PE, the RM, the DSU or the DMA in emission can
be upper-bounded with equation~\ref{eq_execModel_SRAMtrans} when the number of
competitors accessing the same bank is greater than 1. 


\begin{equation}
    \label{eq_execModel_SRAMtrans}
    T_{trans}^{SRAM} ( S_{trans} , N_{comp}^{SRAM} ) = \left\lceil \dfrac{S_{trans}}{W_{bus}^{SRAM}} \right\rceil \times N_{comp}^{SRAM} \times T_{access}^{SRAM}
\end{equation}
with: $W_{bus}^{SRAM}$ the width of the memory bus (8 bytes on the \mppalong) and $N_{comp}^{SRAM}$ a \emph{virtual} number of competitors which depends on the initiator (a PE, the RM or the DMA).

The $N_{comp}^{SRAM}$ value depends on the initiator. For example, a cache miss
from a PE should be considered with $N_{comp}^{SRAM} = 64$. Indeed, the first
Round-Robin arbiter at cache level will, in the worst case, forward to the next
level of arbitration one out of two requests. The second arbiter will forward
one out of sixteen requests and the last level one out of two. Eventually, in
the worst case, one out of sixty-four requests issued to a bank will be
initiated by the PE's cache under consideration. With a similar approach,
$N_{comp}^{SRAM} = 6$ for the RM, the DSU and the DMA Tx.

\begin{example}[Cost of accessing the local SRAM]
    \label{ex_execModel_SRAMaccessTime} Let us assume a PE reading 64 bytes
    from one local SRAM bank. The transaction's address is aligned on a
    multiple of 8 bytes. With $S_{trans} = 64$ and $N_{comp}^{SRAM} = 64$, this
    transaction can take in the absolute worst case $T_{trans}^{SRAM} = 5120$
    clock cycles. If the PE is guaranteed to access the bank without
    competitors, the equation~\ref{eq_execModel_SRAMtrans} does not apply any
    more and the transaction's duration will last for $17$ cycles. Indeed
    $N_{comp}^{SRAM}=1$ and $T_{access}^{SRAM}$ is payed only for the first
    double-word access (the others will follow consecutively without repaying
    for $T_{access}^{SRAM}$ every time).
\end{example}

The only way to \emph{safely} bound SRAM access time of PEs requires to
consider the absolute worst bank-level competition if no assumption can be made
on workloads ran by other PEs. As explained in
Example~\ref{ex_execModel_SRAMaccessTime}, the cost of worst case competition
can be prohibitive.  Moreover, in case of DMA receptions, $T_{trans}^{SRAM}$
should be extended with the maximum stalling time because of DMA Rx
transactions. This stalling time depends on the NoC parameters and the memory
areas assigned to Rx channels. Yet, any bound on the DMA Rx interference
penalty can be summed to $T_{trans}^{SRAM}$ safely thanks to the property of
full timing compositionality of the k1b cores.

Anyway, the cost of bounding SRAM access time can be very large if the
worst-case competition needs to be assumed. Reducing the cost of such an
assumption will be one of the goal of our execution model.


\subsection{WCTT on the NoC} \label{ssec_execModel_WCTTNoC} Bounding traversal
time on the NoC depends both on the packetization procedure and the arbitration
operated in switches. We analyze these two problems in the following
sub-sections.


\subsubsection{Packetization}
With $N_{bytes}^{flit}$ the number of bytes in one NoC flit, the number of 
payload flits needed to send a buffer of $S_{trans}$ bytes over the NoC is:
\begin{displaymath}
    N_{flit} ( S_{trans} ) = 
    \left\lceil \dfrac{S_{trans}}{N_{bytes}^{flit}} \right\rceil
\end{displaymath}

So, assuming $N_{flit}^{pkt}$ to be the maximum number of flits in a NoC packet,
the number of packets required to send $S_{trans}$ bytes is:
\begin{displaymath}
    N_{pkt} ( S_{trans} ) = 
    \left\lceil \dfrac{ N_{flit}(S_{trans}) }{N_{flit}^{pkt}} \right\rceil
\end{displaymath}

Thus, with $N_{flit}^{head}$ the number of flits in a NoC packet header and
$N_{flit}^{bbl}$ the number of empty flits lost in the bubble separating two
consecutive packets, we can derive the total number of flits (including
\emph{wasted} flits lost between packets, if any) needed to send $S_{trans}$
bytes as:

\begin{displaymath}
    N_{flit}^{total} ( S_{trans} ) =
    N_{flit} ( S_{trans} ) +
    N_{pkt} ( S_{trans} ) \times N_{flit}^{head} +
    (N_{pkt} ( S_{trans} ) - 1) \times N_{flit}^{bbl}
\end{displaymath}


\subsubsection{Time required to cross the NoC}
\label{sssec_execModel_NoCbounds}

The time required for all the flits of a transaction of $S_{trans}$ bytes to
cross a NoC route of $N_{switch}$ switches without conflicts is given in
Equation~\ref{eq_execModel_NocTransNoConflict} with $\Delta_L$ the latency of a
NoC link and $\Delta_S$ the latency of a NoC switch.

\begin{equation}
    T_{trans}^{NoC} (S_{trans} , N_{switch}) = 
    (N_{switch} + 1) \times \Delta_L +
    N_{switch}  \times \Delta_S +
    N_{flit}^{total} ( S_{trans} )
    \label{eq_execModel_NocTransNoConflict}
\end{equation}

If we do not make assumptions on the routes or the injection rates of packets,
bounding the WCTT on the NoC \emph{with conflicts} can be impossible because of
the possibility of deadlocks as explained in
Section~\ref{sssec_stateOfTheArt_Noc_deadlocks}. What we emphasize here is the
need for precise assumptions on the NoC access patterns to be able to use it
under real-time constraints.  As an example, we refer the reader to the
contributions of Dupont de Dinechin \etal~\cite{Dinechin2014} and Giannopoulou
\etal~\cite{Giannopoulou2015} for WCTT computation using Network Calculus where
NoC accesses were assumed to be rate-limited at source and to follow
pre-assigned routes.

Overall, bounding WCTTs on the NoC cannot be done without assumption on its
utilization by applications. Providing users with a framework enabling
predictable NoC usage will be the second goal of our execution model.

\subsection{DRAM access time}
\label{ssec_execModel_DRAMaccessTime}
\subsubsection{Model of the arbiter}

The \mppalong's MPFE handles a complex dynamic priority assignment policy to
avoid starvations. For simplification purpose, we will assume a specific MPFE
configuration with starvation counters equal to 0 for all masters. In this
case, all masters always have the same priority (the highest) and are thus
arbitrated in a pure Round-Robin fashion.

With $W_{burst}^{DDR}$ the width of a DDR burst, a transaction of $S_{trans}$
bytes will require the following number of column access requests to be fully
executed:

\begin{displaymath}
    N_{req}^{DDR}(S_{trans}) = \left\lceil \dfrac{S_{trans}}{W_{burst}^{DDR}} \right\rceil
\end{displaymath}

Yet, assuming pure Round-Robin for the MPFE, the number of arbitration rounds
required for all the $N_{req}^{DDR}(S_{trans})$ to be forwarded to the reorder
core is: 

\begin{displaymath}
    N_{round}^{DDR}(S_{trans}, N_{comp}^{DDR}) = N_{req}^{DDR}(S_{trans}) \times N_{comp}^{DDR}
\end{displaymath}


A request entering the reorder core is guaranteed by the hardware to be served
after at most $2 N_{req}^{pool} -1$ other requests. So, with $T_{req}^{DDR}$
the worst case service time for a DDR request, the total amount of time
(without considering refreshes yet) needed to complete a full transaction can
be bounded by Equation~\ref{eq_execModel_DRAMtransTime}.


\begin{equation}
    T_{trans}^{DDR} (S_{trans}, N_{comp}^{DDR}) = 
    (N_{round}^{DDR}(S_{trans}, N_{comp}^{DDR}) + 2 N_{req}^{pool} -1) \times T_{req}^{DDR}
    \label{eq_execModel_DRAMtransTime}
\end{equation}


\subsubsection{Evaluation of $T_{req}^{DDR}$}
As explained in Section~\ref{sssec_stateOfTheArt_basicsDRAM}, a complex access
protocol is used to deal with DDR-SDRAM systems. The service time of a
DDR-SDRAM request highly depends on the current state of the chip, and thus
depends on the history of accesses that changed this state. The worst case
access time of a DRAM request occurs in case of a row-miss read request
following a write request. In this case, the controller must: 1) wait for the
write completion ($t_{WR}$); 2) issue a \emph{PRE} command ($t_{RP}$) on the
currently opened row; 3) activate the correct row ($t_{RCD}$) and 4) issue the
\emph{RD} command ($t_{CAS} + t_{burst}$). The worst case service time for a
request is given in Equation~\ref{eq_execModel_DRAMworstCaseReq}.

\begin{equation}
    T_{req}^{DDR} = 
    t_{WR} +
    t_{RP} +
    t_{RCD} +
    t_{CAS} +
    t_{burst}
    \label{eq_execModel_DRAMworstCaseReq}
\end{equation}


\begin{example}[Cost of accessing the DDR-SDRAM]
    \label{ex_execModel_costAccessDDR}
    Let us assume 192 bytes of data to be written in a bank of the DDR-SDRAM.
    With the values of Table~\ref{table_execModel_MPPAhwParams}, and assuming 4
    DMAs competing for accessing the same bank, $N_{round}^{DDR} = 12$. With
    the DDR timing values of
    Table~\ref{table_stateOfTheArt_exampleTimingParamsMicronDRAM},
    $T_{req}^{DDR} = 67.5ns$. Thus, $T_{trans}^{DDR} \approx 1.8 \mu s$. If
    only one DMA was accessing the bank without any competitor, the row
    activation would need to by payed only once, thus leading to
    $T_{trans}^{DDR} = 75.5ns$.
\end{example}

As explained in Example~\ref{ex_execModel_costAccessDDR}, accessing the
DDR-SDRAM in the worst-case competition scenario can lead to very inefficient
use of the memory. Indeed, competitors accessing different rows of the same
bank will force the controller to repeatedly precharge and activate rows, thus
reducing the time used to effectively transfer data. Reducing the overhead
involved by such competition while keeping bounds on memory transactions sound
will be the third goal of our execution model.


\begin{table}
\begin{center}
    \begin{tabular*}{0.95\linewidth}{@{\extracolsep{\fill}} c c c}
	\hline
        {\sc \textbf{Parameter}} 	& {\sc \textbf{Meaning}} & \textbf{\sc \textbf{Value}} 	\\
	    \hline
         & \emph{SRAM} & \\
        \hline
        $t_{ck}$ & Duration of a clock cycle & $1 / 400$MHz = $2.5$ns \\
        $T_{access}^{SRAM}$ & Duration of a single SRAM access & 10 cycles \\
        $W_{bus}^{SRAM}$ & Width of the SRAM access bus & 8 bytes \\
        $S_{bank}^{SRAM}$ & Size of a local SRAM bank & 128 KiB \\
	    \hline
         & \emph{NoC} & \\
        \hline
        $\Delta_L$ & Latency of a NoC link (not flying) &  See~\cite{kalray_mppa}\\
        $\Delta_S$ & Latency of a NoC switch & See~\cite{kalray_mppa}\\
        $N_{bytes}^{flit}$ & Number of bytes in a NoC flit & 4 bytes \\
        $N_{flit}^{pkt}$ & Number of flits (max.) in a NoC packet & Configurable. <62 \\
        $N_{flit}^{head}$ & Default number of flits in a packet header & 2 \\
	    \hline
         & \emph{DDR3-SDRAM} & \\
        \hline
        $N_{req}^{pool}$ & Size of the reordering pool & 8 requests \\
        $T_{req}^{DDR}$ & Worst case service time for a DRAM request &  Depends on module \\
        $W_{col}^{DDR}$ & Width of a DRAM column / burst size & 64 bytes \\
        \hline	
\end{tabular*}
\end{center}
\caption{Notations and values of the \mppalong hardware parameters}
\label{table_execModel_MPPAhwParams}
\end{table}






\subsection{Summary}

Hosting several applications on a single target while making them independent
from each other requires for each of them to assume worst-case competitors when
accessing shared resources. In the previous sections, we have seen that such an
assumption can lead to very high costs to bound the durations of accesses to
the local SRAM, the NoC or the DDR-SDRAM. In order to keep co-hosting costs
reasonable, we propose to artificially isolate applications by enforcing
specific behaviours on shared resources. By mastering concurrent applications
at the resource level, we aim at avoiding the corner cases for which accessing
resources is too expensive, thus enabling better usage of the hardware overall.


\section{Mitigation of the interferences}

In order to mitigate the interferences when accessing shared resources, we
propose an execution model which constrains the behaviour of applications in
exchange of improving their predictability and enabling better access to shared
resources. The idea is to provide temporally isolated environments, denoted as
\emph{partitions}, to applications. 

\begin{definition}[Partition]
A partition is an execution environment in which the temporal behaviour of an
    application does not depend on the behaviour of applications not belonging
    to the same partition. 
\end{definition}

Our execution model imposes specific access patterns to sensible resources that
are likely to be shared by several partitions in order to enforce
interference-free executions of the isolated applications. By doing so, we aim
at executing partitions in a \emph{time-composable} manner.


\subsection{Definition of the execution model}
We propose four rules constraining accesses to the three main levels where
interferences can occur on the \mppalong, namely the local SRAM, the NoC and
the DDR-SDRAM.

\newtheorem{emrule}{Rule}
\begin{emrule}
    Any PE (respectively any local SRAM bank) inside any compute cluster can be
    used by at most one partition.  \label{emrule_1}
\end{emrule}

\begin{emrule}
    Communications over the NoC must respect a pre-computed Time-Division
    Multiplexing schedule and avoid conflicts by either taking non-overlapping
    routes or accessing the shared route at different time. The NoC schedule
    must ensure that when a communication uses a route, it is completely
    reserved at the time of utilization.  \label{emrule_2}
\end{emrule}

\begin{emrule}
    The memory areas to be sent over the NoC must be defined off-line.
    \label{emrule_3}
\end{emrule}

\begin{emrule}
    Any bank of the external DDR-SDRAM can be shared by two or more partitions
    if and only if they never access it simultaneously.  \label{emrule_4}
\end{emrule}

\subsection{Motivations}
\subsubsection{Rule~\ref{emrule_1}}
With Rule~\ref{emrule_1}, we ensure that some code running on a PE will not
suffer from local SRAM interferences caused by PEs of other partitions. By
doing so, the $N_{comp}^{SRAM}$ parameter of
Equation~\ref{eq_execModel_SRAMtrans} can be adjusted to account only for the
PEs generating \emph{friend} traffic, that is, PEs running code of the same
partition. This does not only help to lower the local SRAM interference
penalties when computing WCETs, but this also brings complete isolation between
partitions.  Yet, local SRAM accesses from the RM and the DMA can still be
conflicting with the PEs. DMA accesses will occur only to send or receive data
that belongs to a partition. Conflicts with PEs of the same partition do not
break the property of temporal isolation. Similarly, as we will detail in
Chapter~\ref{chap_implemExecMod}, the RM will be executing a custom hypervisor
which may access partition-reserved memory addresses only to achieve
partition-related work.

\begin{example}[Application of Rule~\ref{emrule_1}]
    An example of application of the Rule~\ref{emrule_1} is shown in
    Figure~\ref{fig_execModel_exampleRule1}. A simplified compute cluster
    architecture is presented in Figure~\ref{fig_execModel_exampleRule1_before}
    with 4 PEs, 4 banks of memory, the RM and the DMA.
    Figure~\ref{fig_execModel_exampleRule1_after} shows the same cluster when
    the Rule~\ref{emrule_1} is applied to split the compute cluster into two
    partitions. PEs 0 and 1 are not able to access the banks 2 and 3 allocated
    to the partition 2. Symmetrically, PEs 2 and 3 cannot access banks 0 and 1
    reserved for the partition 1. In this case, any task running on the PE 0
    (for example) may suffer only from ``friend'' memory interferences when in
    competition with the PE 1, the RM or the DMA. In this example, the
    partitioning has arbitrarily been chosen as symmetric (2 cores - 2 banks
    per partition) but any configuration could have been possible (1 core - 3
    banks, ...) depending on the needs of applications.
\end{example}

\begin{figure}
    \centering
    \begin{subfigure}[b]{0.45\linewidth}
    \centering
        \scalebox{0.7}{

% 1 : x; 2: y; 3: tikz object name
\newcommand\smemRRarb[3]{
    \node[circle, draw, color=black, anchor=south, thick, minimum size=1cm, inner sep=0pt] (#3) at (#1, #2) {A};
    \draw[ultra thick, ->, >=latex] (#3.west) -- ([yshift=-0.6em]#3.west); 
}

% X, Y, PRINTNAME, tikzlabel
\newcommand\smemCore[4]{
    \node[rectangle, draw, color=black, thick, anchor=south west, minimum width=2cm, minimum height=1cm, inner sep=0pt] (#4) at (#1, #2) {#3};
}

% X, Y, PRINTNAME, tikzlabel
\newcommand\smemBank[4]{
    \node[rectangle, draw, color=black, thick, anchor=south west, minimum width=2.5cm, minimum height=1cm, inner sep=0pt] (#4) at (#1, #2) {\emph{#3}};
    \smemRRarb{#1-1}{#2}{arb#4}
    \draw[thick] (arb#4.east) -- (#4.west);
}


\begin{tikzpicture}[font={\fontsize{12pt}{12}\selectfont}]

    \smemCore{0}{-6}{RM}{rm}
    \smemCore{0}{1.5}{DMA}{dma}
    
    \smemCore{0}{0}{PE 0}{pe0}
    \smemBank{6}{0}{Bank 0}{bank0}
        
    \smemCore{0}{-1.5}{PE 1}{pe1}
    \smemBank{6}{-1.5}{Bank 1}{bank1}
    
    \smemCore{0}{-3}{PE 2}{pe2}
    \smemBank{6}{-3}{Bank 2}{bank2}
    
    \smemCore{0}{-4.5}{PE 3}{pe3}
    \smemBank{6}{-4.5}{Bank 3}{bank3}

    \draw[thick] (pe0.east) -- (arbbank0.west);
    \draw[thick] (pe0.east) -- (arbbank1.west);
    \draw[thick] (pe0.east) -- (arbbank2.west);
    \draw[thick] (pe0.east) -- (arbbank3.west);

    \draw[thick] (pe1.east) -- (arbbank0.west);
    \draw[thick] (pe1.east) -- (arbbank1.west);
    \draw[thick] (pe1.east) -- (arbbank2.west);
    \draw[thick] (pe1.east) -- (arbbank3.west);

    \draw[thick] (pe2.east) -- (arbbank0.west);
    \draw[thick] (pe2.east) -- (arbbank1.west);
    \draw[thick] (pe2.east) -- (arbbank2.west);
    \draw[thick] (pe2.east) -- (arbbank3.west);

    \draw[thick] (pe3.east) -- (arbbank0.west);
    \draw[thick] (pe3.east) -- (arbbank1.west);
    \draw[thick] (pe3.east) -- (arbbank2.west);
    \draw[thick] (pe3.east) -- (arbbank3.west);

    \draw[thick, densely dotted] (dma.east) -- (arbbank0.west);
    \draw[thick, densely dotted] (dma.east) -- (arbbank1.west);
    \draw[thick, densely dotted] (dma.east) -- (arbbank2.west);
    \draw[thick, densely dotted] (dma.east) -- (arbbank3.west);
    
    \draw[thick, densely dotted] (rm.east) -- (arbbank0.west);
    \draw[thick, densely dotted] (rm.east) -- (arbbank1.west);
    \draw[thick, densely dotted] (rm.east) -- (arbbank2.west);
    \draw[thick, densely dotted] (rm.east) -- (arbbank3.west);

\end{tikzpicture}
}
        \caption{Without Rule~\ref{emrule_1}}
        \label{fig_execModel_exampleRule1_before}
    \end{subfigure}
    \begin{subfigure}[b]{0.45\linewidth}
    \centering
        \scalebox{0.7}{

% 1 : x; 2: y; 3: tikz object name
\newcommand\smemRRarb[3]{
    \node[circle, draw, color=black, anchor=south, thick, minimum size=1cm, inner sep=0pt] (#3) at (#1, #2) {A};
    \draw[ultra thick, ->, >=latex] (#3.west) -- ([yshift=-0.6em]#3.west); 
}

% X, Y, PRINTNAME, tikzlabel
\newcommand\smemCore[4]{
    \node[rectangle, draw, color=black, thick, anchor=south west, minimum width=2cm, minimum height=1cm, inner sep=0pt] (#4) at (#1, #2) {#3};
}

% X, Y, PRINTNAME, tikzlabel
\newcommand\smemBank[4]{
    \node[rectangle, draw, color=black, thick, anchor=south west, minimum width=2.5cm, minimum height=1cm, inner sep=0pt] (#4) at (#1, #2) {\emph{#3}};
    \smemRRarb{#1-1}{#2}{arb#4}
    \draw[thick] (arb#4.east) -- (#4.west);
}


\begin{tikzpicture}[font={\fontsize{12pt}{12}\selectfont}]

    \smemCore{0}{-6}{RM}{rm}
    \smemCore{0}{1.5}{DMA}{dma}
    
    \smemCore{0}{0}{PE 0}{pe0}
    \smemBank{6}{0}{Bank 0}{bank0}
        
    \smemCore{0}{-1.5}{PE 1}{pe1}
    \smemBank{6}{-1.5}{Bank 1}{bank1}
    
    \smemCore{0}{-3}{PE 2}{pe2}
    \smemBank{6}{-3}{Bank 2}{bank2}
    
    \smemCore{0}{-4.5}{PE 3}{pe3}
    \smemBank{6}{-4.5}{Bank 3}{bank3}

    \draw[thick] (pe0.east) -- (arbbank0.west);
    \draw[thick] (pe0.east) -- (arbbank1.west);

    \draw[thick] (pe1.east) -- (arbbank0.west);
    \draw[thick] (pe1.east) -- (arbbank1.west);

    \draw[thick] (pe2.east) -- (arbbank2.west);
    \draw[thick] (pe2.east) -- (arbbank3.west);

    \draw[thick] (pe3.east) -- (arbbank2.west);
    \draw[thick] (pe3.east) -- (arbbank3.west);

    \draw[thick, densely dotted] (dma.east) -- (arbbank0.west);
    \draw[thick, densely dotted] (dma.east) -- (arbbank1.west);
    \draw[thick, densely dotted] (dma.east) -- (arbbank2.west);
    \draw[thick, densely dotted] (dma.east) -- (arbbank3.west);
    
    \draw[thick, densely dotted] (rm.east) -- (arbbank0.west);
    \draw[thick, densely dotted] (rm.east) -- (arbbank1.west);
    \draw[thick, densely dotted] (rm.east) -- (arbbank2.west);
    \draw[thick, densely dotted] (rm.east) -- (arbbank3.west);

    \node[rectangle, draw, color=black, thick, dashed, anchor=south west, minimum width=8.8cm, minimum height=2.8cm, inner sep=0pt] at (-0.15, -1.65) { };
    \node[minimum width=2.5cm, anchor=south west] at (6,1.3) {\Large Part. 1};

    \node[rectangle, draw, color=black, thick, dashed, anchor=south west, minimum width=8.8cm, minimum height=2.8cm, inner sep=0pt] at (-0.15, -4.65) { };
    \node[minimum width=2.5cm, anchor=north west] at (6,-4.8) {\Large Part. 2};

\end{tikzpicture}
}
        \caption{With Rule~\ref{emrule_1}}
        \label{fig_execModel_exampleRule1_after}
    \end{subfigure}
    \caption{Example of application of the Rule~\ref{emrule_1} of the execution model on simplified compute cluster architecture with 2 partitions}
    \label{fig_execModel_exampleRule1}
\end{figure}

\subsubsection{Rule~\ref{emrule_2}}
Enforcing a Time-Driven schedule of the NoC enables an easy and efficient
estimation of WCTTs of packets. Indeed, assuming that no conflict can occur on
the NoC, the traversal time of packets can be estimated accurately using simple
models such as Equation~\ref{eq_execModel_NocTransNoConflict}. However, these
gains are obtained at the costs of inflexible communication scenarios and
additional efforts for pre-computing schedules that actually exhibit the
no-overlapping property.

The approaches considering asynchronous and rate-limited sources such as
\cite{Dinechin2014} and \cite{Giannopoulou2015} are often argued to provide
more flexibility and good guaranteed bandwidths. However, our choice of
considering a TDM schedule and synchronous sources over Network Calculus and
asynchronous sources is motivated by the two following reasons:

\begin{enumerate}
    \item We aim at providing fully temporally isolated execution environments
        to partitions. Approaches with asynchronous sources often rely on the
        timing \emph{compositionality} of the system to account for the worst
        case communication conflict. By doing so, the actual time required for
        a packet to cross the network, even if safely bounded, does depend on
        the behaviour of the co-running applications. By using a TDM schedule,
        we aim at providing a property of timing \emph{composability} which is
        better suited to ensure strict isolation.
    \item We aim at being able to efficiently execute multi-cluster
        applications in multi-cluster partitions to fully benefit from the
        massively parallel computational power of the target. To do so, our
        ability to ensure short guaranteed \emph{latencies} for data exchange
        between clusters will be of major importance. TDM schedules have been
        shown in~\cite{Puffitsch2015} to provide better results for this
        particular criteria while Network Calculus on asynchronous sources
        seems preferable when \emph{bandwidth} guarantees are needed.
\end{enumerate}

\begin{example}[Application of Rule~\ref{emrule_2}]
    A simplified NoC topology with 6 nodes is depicted in
Figure~\ref{fig_execModel_exampleRule2_noc} where two communications compete to
access the \circled{2} $\to$ \circled{5} NoC resource.
Figure~\ref{fig_execModel_exampleRule2_diagram} shows an example of correct TDM
schedule of the two communications where temporal partitioning ensures the
conflicts avoidance.  \end{example}

\begin{figure}
    \centering
    \begin{subfigure}[b]{0.4\linewidth}
    \centering
        \scalebox{0.7}{
% 1 : x; 2: y; 3: tikz object name
\newcommand\nocnode[3]{
    \node[circle, draw, color=black, anchor=south, thick, minimum size=1cm, inner sep=0pt] (r#3) at (#1, #2) {#3};
}

\begin{tikzpicture}[font={\fontsize{12pt}{12}\selectfont}]
    
    \nocnode{0}{0}{1}
    \nocnode{3}{0}{2}
    \nocnode{6}{0}{3}
    \nocnode{0}{-2.5}{4}
    \nocnode{3}{-2.5}{5}
    \nocnode{6}{-2.5}{6}

    \draw[-latex, thick, loosely dashed] (r1) -- (r2);
    \draw[-latex, thick, loosely dashed] (r2.230) -- (r5.130);
    \draw[-latex, thick, loosely dashed] (r5) -- (r6);

    \draw[-latex, thick, densely dotted] (r3) -- (r2);
    \draw[-latex, thick, densely dotted] (r2.310) -- (r5.50);
    \draw[-latex, thick, densely dotted] (r5) -- (r4);

\end{tikzpicture}
}
        \caption{Simplified NoC}
        \label{fig_execModel_exampleRule2_noc}
    \end{subfigure}
    \begin{subfigure}[b]{0.59\linewidth}
    \centering
        \tikzset{timing/.append style={x=1.8ex, y=3ex}}
        

\begin{tikztimingtable}[timing/lslope=0, timing/slope=0, timing/coldist=0.5, timing/d/text/.append style={font=\rmfamily}, timing/d/background/.style={fill=gray}]
    $1 \to 6$ & L 3D 6L 3D 6L \\     
    $3 \to 4$ & 5L 2D 7L 2D 3L \\     
\extracode
\begin{background}
    \vertlines[dashed]{}
\end{background}
\end{tikztimingtable}



        \caption{Communication schedule}
        \label{fig_execModel_exampleRule2_diagram}
    \end{subfigure}
    \caption{Example of application of the Rule~\ref{emrule_2} of the execution model on simplified NoC with 2 communications competing for a shared NoC resource \circled{2} $\to$ \circled{5}}
    \label{fig_execModel_exampleRule2}
\end{figure}


\subsubsection{Rule~\ref{emrule_3}}
Although meeting this rule is not absolutely necessary to ensure full temporal
isolation, it has several interesting qualities. Firstly, knowing a priori the
buffers that will be sent during the strictly periodic slot imposed by
Rule~\ref{emrule_2} enables to \emph{verify} off-line the coherency of a
partition. The size of the buffer to be sent can be compared with the width of
the TDM slots using Equation~\ref{eq_execModel_NocTransNoConflict} to check
whether the time slot is sufficiently large for all the data to be sent.
Secondly, as we will show in Chapter~\ref{chap_implemExecMod},
Rule~\ref{emrule_3} greatly simplifies the implementation of the hypervisor in
charge of enforcing the respect of the execution model. In particular, it helps
in narrowing the WCET of the hypervisor which we will demonstrate to be of
major importance for performance issues. And finally, knowing where and when
DMA Rx transactions will occur enables accurate estimations of the stalling
time that PEs will suffer when accessing their local SRAM banks, thus solving
the problem identified in Section~\ref{ssec_execModel_SRAMaccessModel}.



\subsubsection{Rule~\ref{emrule_4}}
With Rule~\ref{emrule_4} we eliminate the possibility of DDR-SDRAM masters
conflicting to access different rows of the same bank. Thanks to this, the
estimation of $T_{req}^{DDR}$ can be adapted to take into account that a master
issuing a series of requests targeting contiguous memory areas in the same row
will have to pay for the activation of this row only for the \emph{first}
request and not for \emph{all} of them. This provides both performance
isolation by exploiting the parallelism of banked DDR-SDRAM but also improves
the overall memory throughput by helping the memory controller issue
inexpensive requests which will not change rows inside banks due to competitors
accessing other rows .

\begin{example}[Application of Rule~\ref{emrule_4}]
    Figure~\ref{fig_execModel_exampleRule4_ddr} shows an example of 3
    partitions $P_A$, $P_B$ and $P_C$ accessing the external DDR-SDRAM composed
    of 6 banks $b_0 , \ldots , b_5$. In particular, one may note that the bank
    $b_2$ is shared by $P_A$ and $P_B$ while $b_3$ is shared by $P_B$ and
    $P_C$. Figure~\ref{fig_execModel_exampleRule4_diagram} shows an example of
    correct schedule of memory transaction where the accesses of $P_A$ and
    $P_C$ are allowed to overlap in time since they do not share any bank. On
    the other hand, the transaction from $P_B$ are never overlapping with $P_A$
    and $P_C$ to ensure competition-free accesses to banks $b_2$ and $b_3$.
\end{example}

\begin{figure}
    \centering
    \begin{subfigure}[b]{0.3\linewidth}
    \centering
        \scalebox{0.6}{
% 1 : x; 2: y; 3: tikz object name; 4: printname
\newcommand\partnode[4]{
    %\node[circle, draw, color=black, thick, minimum size=1cm, inner sep=0pt] (#3) at (#1, #2) {#4};
    \node[rectangle, draw, rounded corners, color=black, thick, minimum width=1cm, minimum height=1cm, inner sep=0pt] (#3) at (#1, #2) {#4};
}

\begin{tikzpicture}[font={\fontsize{12pt}{12}\selectfont}]
    
    \node[circle, draw, color=black, thick, minimum size=1.3cm, inner sep=0pt] (arb) at (0, 0) {Arb.};
    \draw[ultra thick, ->, >=latex] (arb.west) -- ([yshift=-0.6em]arb.west); 
   
    \partnode{-2}{-1.2}{A}{$P_A$}
    \partnode{0}{-1.6}{B}{$P_B$}
    \partnode{2}{-1.2}{C}{$P_C$}

    \draw[-latex, thick] (A) -- (arb);
    \draw[-latex, thick] (B) -- (arb);
    \draw[-latex, thick] (C) -- (arb);
    
    \node[rectangle, draw, color=black, thick, dashed, anchor=south west, minimum width=7.4cm, minimum height=3cm] (ddrmodule) at (-3.7, 1.2) {};
    \node[anchor=north west, minimum height=0.5cm] at (-3.7, 1.2) {\emph{\small DDR-SDRAM}};
    \draw[-latex, thick] (arb.north) -- (ddrmodule.south);

    \node[anchor=north west, minimum width=1cm, minimum height=0.7cm, ] at (-3.5, 2) {$b_0$};
    \node[rectangle, draw, color=black, pattern=north east lines, thick, anchor=south west, minimum width=1cm, minimum height=2cm, ] at (-3.5, 2) {}; %B0
    
    \node[anchor=north west, minimum width=1cm, minimum height=0.7cm, ] at (-2.3, 2) {$b_1$};
    \node[rectangle, draw, color=black, pattern=north east lines, thick, anchor=south west, minimum width=1cm, minimum height=2cm] at (-2.3, 2) {}; %B1
    
    \node[anchor=north west, minimum width=1cm, minimum height=0.7cm, ] at (-1.1, 2) {$b_2$};
    \node[rectangle, draw, color=black, pattern=north east lines, thick, anchor=south west, minimum width=1cm, minimum height=0.6cm] at (-1.1, 2) {}; %B2
    \node[rectangle, draw, color=black, pattern=dots, thick, anchor=south west, minimum width=1cm, minimum height=1.4cm] at (-1.1, 2.6) {}; %B2
    
    \node[anchor=north west, minimum width=1cm, minimum height=0.7cm, ] at (0.1, 2) {$b_3$};
    \node[rectangle, draw, color=black, pattern=dots, thick, anchor=south west, minimum width=1cm, minimum height=1.3cm] at (0.1, 2) {}; %B3
    \node[rectangle, draw, color=black, pattern=grid, thick, anchor=south west, minimum width=1cm, minimum height=0.7cm] at (0.1, 3.3) {}; %B3
    
    \node[anchor=north west, minimum width=1cm, minimum height=0.7cm, ] at (1.3, 2) {$b_4$};
    \node[rectangle, draw, color=black, pattern=grid, thick, anchor=south west, minimum width=1cm, minimum height=2cm] at (1.3, 2) {}; %B4
    
    \node[anchor=north west, minimum width=1cm, minimum height=0.7cm, ] at (2.5, 2) {$b_5$};
    \node[rectangle, draw, color=black, pattern=grid, thick, anchor=south west, minimum width=1cm, minimum height=2cm] at (2.5, 2) {}; %B5

    
    \node (noteA) at (-4,3.5) {$P_A$};
    \draw (noteA) -- (-3,3);

    \node (noteB) at (-1,4.5) {$P_B$};
    \draw (noteB) -- (-0.5,3.5);

    \node (noteC) at (2.5,4.5) {$P_C$};
    \draw (noteC) -- (2,3.5);
\end{tikzpicture}
}
        \caption{DDR-SDRAM accesses}
        \label{fig_execModel_exampleRule4_ddr}
    \end{subfigure}\hspace{6mm}
    \begin{subfigure}[b]{0.65\linewidth}
    \centering
        \tikzset{timing/.append style={x=2.2ex, y=3ex}}
        

\begin{tikztimingtable}[timing/lslope=0, timing/slope=0, timing/coldist=0.5, timing/d/text/.append style={font=\rmfamily}, timing/d/background/.style={fill=gray}]
    $P_A$ & L 3D 7L 3D 5L \\     
    $P_B$ & 5L D 9L D 3L \\     
    $P_C$ & 2L 1.5D 3.5L 1.5D 3.5L 1.5D 3.5L 1.5D 0.5L \\     
\extracode
\begin{background}
    \vertlines[dashed]{}
\end{background}
\end{tikztimingtable}



        \caption{DDR-SDRAM access schedule}
        \label{fig_execModel_exampleRule4_diagram}
    \end{subfigure}
    \caption{Example of application of the Rule~\ref{emrule_4} of the execution model on simplified DDR-SDRAM arbiter with 3 masters $P_A$, $P_B$ and $P_C$ competing to access the memory}
    \label{fig_execModel_exampleRule4}
\end{figure}

\section{Summary}
In this chapter, we analyzed the architecture of the \mppalong in details with
a particular focus on arbitration mechanisms for the access to shared
resources. Based on this analysis, we proposed, when it was possible,
mathematical expressions to compute worst-case interference penalties for the
access to the local SRAM banks, the NoC and the DDR-SDRAM. We showed that, in
most cases, considering the pathological worst-case pattern of accesses to the
resources leads to prohibitive penalties while, in the remaining cases,
computing such bounds was simply not possible without eliminating the
worst-case thanks to assumptions on the system model. Based on these
conclusions, we proposed four rules composing an execution model that mitigates
or even eliminates undesired interferences between applications. This execution
model relies on pure spatial partitioning inside compute clusters and on hybrid
partitioning for the NoC and the DDR-SDRAM to provide temporally isolated
\emph{partitions} in which applications can run. Overall, our approach relies
on off-line reservation of resources by partitions. We formalize this using a
notion of \emph{resource budget} that is allocated to a partition. Budgets are
key elements for the framework proposed in this thesis. This first contribution
has been published in~\cite{Perret16}. The next chapter provides a formal
definition of the notions of partitions and budgets and exposes the workflow
relying on it.



\clearpage
\subbiblio
\end{document}
